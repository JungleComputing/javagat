\section{Server Implementation}
Below we describe our (advert) server in more detail.

\subsection{User Authentication}
As described in the Server Design we will implement two servers, one without user
authentication (i.e. a public server without any guarantees), and a server which
requires authentication through Google Accounts.

\subsubsection{Google Accounts}
The most basic and obvious way to achieve authentication is single user
authentication (through Google Accounts). Basically only the owner is allowed to
use the advert service. According to the documentation provided by Google, there
is no variable that indicates if a user is owner of the application. Nonetheless
there is another function called \texttt{is\_current\_user\_admin()}, which is
available in the \texttt{google.appengine.api.users} package. By default the
owner of the application is administrator, and additional administrators can be
added through the administration panel. This fits our needs perfectly.

\dashline{
Once a client logged in through the Google login procedure (e.g. a Google login
page), the request handler at the server returns a user object, as shown in
Figure \ref{serverimpl-auth}. This object can then be used to identify a
user.
}

\begin{figure*}[ht] %[placement] where placement is h,t,b,p
\begin{center}
\begin{code}
user   = users.get_current_user()

if not user:
    self.error(403)
    self.response.headers['Content-Type'] = 'text/plain'
    self.response.out.write('Not Authenticated')
    return

if not users.is_current_user_admin():
    self.error(403)
    self.response.headers['Content-Type'] = 'text/plain'
    self.response.out.write('No Administrator')
    return
    
...
\end{code}
\caption{Authenticating a User.\label{serverimpl-auth}}
\end{center}
\end{figure*}

Other functions included in the \texttt{google.appengine.api.users} package are:

\begin{itemize} 
\item \texttt{create\_login\_url(dest\_url)}; which returns a URL that, when
visited, will prompt the user to sign in using a Google account, then redirect
the user back to the URL given as \texttt{dest\_url}.
\item \texttt{create\_logout\_url(dest\_url)}; which returns a URL that, when
visited, will sign the user out, then redirect the user back to the URL given as
\texttt{dest\_url}.
\end{itemize}

\subsection{The Datastore}
\subsubsection{Datastore Layout}
The Google App Engine datastore is not like a traditional relational database.
Data objects, or ``entities'', have a kind and a set of properties. For our
server we need a set of different data objects to store in our datastore.

\paragraph{Storing (Advert) Objects}
Obviously, the main purpose of our server is to store binary (advert) objects.
For this purpose we designed the following entity to represent our advert data
(Figure \ref{serverimpl-ds}). First of all we will store a path, which will be
used for reference of a unique object (i.e. a key), which will be structured as a
directory structure. Secondly, we will store the author's name, for legal issues.
Next is the actual Advert object, which is stored as binary data. Finally we add
a TTL as will be discussed below.

\begin{figure*}[ht] %[placement] where placement is h,t,b,p
\begin{center}
\begin{code}
class Advert(db.Model):
  path   = db.StringProperty()
  author = db.UserProperty()
  ttl    = db.DateTimeProperty(auto_now_add=True)
  object = db.BlobProperty()
  
class MetaData(db.Model):
  keystr = db.StringProperty()
  value  = db.StringProperty()
\end{code}
\caption{An Advert Object.\label{serverimpl-ds}}
\end{center}
\end{figure*}

\paragraph{Storing Meta Data}
At our server, we would like to store meta data not as a binary object, but as
readable strings of data. This way, we can search meta data when it is queried.
According to the AdvertService API, a MetaData object should look like Figure
\ref{serverimpl-ds}. In this object, we just need to store key-value pairs,
which are String properties. Every object will be linked to a parent (Advert
object), as soon as it's created.

\subsubsection{Path Encoding}
To have the AdvertService work properly, we need some path encoding to save
Advert objects in the datastore. The Google App Engine itself provides one
solution, because every entity in the datastore has a key. When the application
creates an entity, it can assign another entity as the parent of the new entity.
Assigning a parent to a new entity puts the new entity in the same entity group
as the parent entity. Every entity belongs to an entity group, a set of one or
more entities that can be manipulated in a single transaction. An entity without
a parent is a root entity. An entity that is a parent for another entity can also
have a parent. A chain of parent entities from an entity up to the root is the
path for the entity, and members of the path are the entity's ancestors. The
parent of an entity is defined when the entity is created, and cannot be changed
later. The downside of this solution is that we have to create empty MetaData
objects for path nodes that are empty.

Another option is to save our own paths, using a namespace that identifies a
path, something like the UNIX system does (e.g. \texttt{/home/bboterm/.}). The
advantage of this model is that it works exactly analogue to the JavaGAT
AdvertService. The disadvantage of this is that we have to parse our own
pathnames, and keep track of them by means of a non-existing mechanism (i.e.
something that we need to build ourselves). Since this is not a major obstacle,
we will stick with the second approach.

\subsection{Public Server Functions}
Below we will list some of the major functions the advert server will offer to
the client. These functions will be used by, for example, the JavaGAT
AdvertService adaptor for the App Engine.

\subsubsection{Logging In}
Since we are using ClientLogin service (as described above) to authenticate
ourselves to the App Engine, we won't need a separate login page. Sending our
credentials (session IDs) to the Appe Engine should be enough to authenticate
ourselves. For every request, our server will use a small section of code, which
looks like the code segment stated in Figure \ref{serverimpl-login}, to
authenticate the user.

\begin{figure*}[ht] %[placement] where placement is h,t,b,p
\begin{center}
\begin{code}
class AnyClass(webapp.RequestHandler):
  def get(self):
    user = users.get_current_user()

    if user and users.is_current_user_admin():
      self.response.headers['Content-Type'] = 'text/plain'
      self.response.out.write('Hello, ' + user.nickname())
    else:
      self.response.http_status_message(401)
      self.response.out.write('Some error.')
\end{code}
\caption{A MetaData Object.\label{serverimpl-login}}
\end{center}
\end{figure*}
      
\subsubsection{Receiving and Storing Binary Data}
An important function of the Advert Service is to accept and store binary data.

\paragraph{Binary Data Only}
For accepting just binary data, we can receive the entire message body and write
it to a variable, which is stored in the datastore accordingly. A sample code of
this function is shown in the code segment of Figure \ref{serverimpl-download}.
The first line identifies the class, which is called by the RequestHandler
(depending on the given URL). The second line says that we are expecting a POST
request. Then, a new \texttt{Bin()} data model is created (which in our case only
contains a field \texttt{data = db.BlobProperty()}). Then we fetch the body and
put in a temporary variable, before we store it as a \texttt{db.BlobProperty()}in
the database by calling \texttt{bin.put()}.

\begin{figure*}[ht] %[placement] where placement is h,t,b,p
\begin{center}
\begin{code}
class Download(webapp.RequestHandler):
  def post(self):
    bin = Bin()
    uploaded_file = self.request.body
    bin.data = db.Blob(uploaded_file)
    bin.put()
    self.redirect('/')
\end{code}
\caption{Accepting Binary Data.\label{serverimpl-download}}
\end{center}
\end{figure*}
      
\paragraph{Combination of Binary Data and Strings}
Although meta data will be stored in a different class, we won't apply a
different function to store MetaData objects, for the simple reason that if one
of these transfers would fail, it could lead to inconsistencies (see Section
\ref{clientimpl-sending-both}).

When receiving our binary object as a combination of both binary data and
unicode Strings, it will be received as one stream of bytes (assuming we're not
using the multipart/formdata method of Section \ref{clientimpl-sending-both}.
This means that we will have to dismember the message ourselves, which is
not as straightforward as it seems. The Google App Engine currently only
supports Python 2.5, which does not support bytes or byte arrays for data
manipulation.

We still managed to extract data using string manipulation. Essentially when the
App Engine receives the entire body as shown in Figure \ref{serverimpl-download},
it is stored as a raw string (no encoding) before it is stored as BLOB in the
datastore. We made use of that raw string as shown in Figure
\ref{serverimpl-raw-string}. In this example, we receive the message body and
store it into the raw string called `bytes'. After this we read the first byte
(\texttt{bytes[0:4]}), we use \texttt{ord()} to make it an integer, knowing the
length of the data sent. After that we send our Content-Type headers and write
the rest of the body to the standard output.

\begin{figure*}[ht] %[placement] where placement is h,t,b,p
\begin{center}
\begin{code}
bytes = self.request.body
length = ord(bytes[0:4])
self.response.headers['Content-Type'] = "image/gif"
self.response.out.write(bytes[5:length])
\end{code}
\caption{Manipulating a Raw String.\label{serverimpl-raw-string}}
\end{center}
\end{figure*}

\paragraph{Simplejson}
\label{serverimpl-simplejson}
\dashline{
Instead of using our own representation of unicode Strings and binary data, we
could also use a JSON representation. Once a serial JSON array or object has
been sent (see Section \ref{clientimpl-sending-both}), we are able to load it
into a server-side JSON object and decode it according to Figure
\ref{serverimpl-json}. The \texttt{simplejson.loads()} function decodes a
serial JSON String representation into a JSON array, after which we can access
its array entries like a regular array. 
} 

\begin{figure*}[ht] %[placement] where placement is h,t,b,p
\begin{center}
\begin{code}
body = self.request.body
json = simplejson.loads(body)
...
advert.path   = json[0] #extract path from message
advert.object = json[2] #extract (base64) object from message
...
for k in json[1].keys():
  ...
  metadata.keystr = k
  metadata.value  = json[1][k]
...
\end{code}
\caption{Decoding a JSON object.\label{serverimpl-json}}
\end{center}
\end{figure*}

\dashline{
The same goes for a JSON object. We can construct a for loop like shown in
Figure \ref{serverimpl-json}, by iterating through all the keys and retrieving
their key-value pairs.
}

\dashline{
For reasons mentioned in Section \ref{clientimpl-sending-both}, the object is
not send in binary, but in unicode String format. For this reason, we can't
store it as a \texttt{db.BlobProperty()}, since this property expects unencoded
binary data only. The only option is to use the \texttt{db.TextProperty()},
since it is not limited to 500 characters, like the \texttt{db.StringProperty()}
is.
}

\subsubsection{Storing Meta Data}
Since MetaData needs to be searchable at the server-side, we will have to extract
the key-value pairs from the POST request and store them in the datastore
accordingly. This will be done in the opposite way of sending them at the Client
side (see Section \ref{clientimpl-sending-both}).

As of yet, the Google App Engine does not support Tuples, or key-value pairs as a
data type. Therefore we created our own MetaData class as described above. Also,
this MetaData cannot be stored into a list, because lists only supports primary
data types like integer and string. For that matter, we add an ID to the
MetaData, which is a child of the binary object stored in the datastore.

Another option would be to maintain two lists of strings, where the indexes of
the lists link the key and the value of the MetaData. However, we are not sure of
those lists maintain order (which could mess up the MetaData).

Finally, we could also append all data in two strings and have them separated by
special delimiter. Downside of this method is that we could run out of the
maximum length of strings (which is 500 bytes).

\subsubsection{Finding MetaData}
When the MetaData has been stored in the datastore, we should be able to process
find() requests by clients. First a MetaData object is received, which is then
tried to be matched with MetaData already present in the datastore. For this
purpose we will use the GQL as described above.
%TODO: <something about GQL queries>

The return value of all metadata found is a String array, which again needs to be
formatted in a way that it can be sent as one String. 
%TODO: <more to follow>

\subsubsection{Returning an Object from the Datastore}
This function will send binary data back to client. Basically, this function only
sends an Object to the standard output (the HTTP connection), which looks like
Figure \ref{serverimpl-bin-response}. Of Course, we first need to fetch the
requested object from the datastore, which will be done using GQL Queries. 
%TODO: <more to follow>

\begin{figure*}[ht] %[placement] where placement is h,t,b,p
\begin{center}
\begin{code}
self.response.headers['Content-Type'] = "application/octet-stream"
self.response.out.write(bin.data)
\end{code}
\caption{HTTP Response with Binary Data.\label{serverimpl-bin-response}}
\end{center}
\end{figure*}

\subsection{Private Server Functions}
Below we will describe some of the private server functions provided for the
advert service.

\subsubsection{Garbage Collector}
As described above, we will use some form of TTL to determine whether a data item
stored is still needed in the datastorage. When a data item is expired, it will
be removed automatically using a mechanism called the Garbage Collector.
%TODO: <more to follow>

\subsubsection{Data Matching}
For our \texttt{find()} function, we will need some internal `match' function,
which can query the database to see if the specified data item is present in the
datastore. For this matching, we will have an internal function which is able to
match data received from the client with data present in the datastore. 
%TODO: <more to follow>


