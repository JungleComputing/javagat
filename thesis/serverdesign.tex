\section{Server Design}
Below we will describe some high-level design decisions we made, prior to
implementing our server on the Google App Engine.

\subsection{Server Structure}
The server will have an iterative structure and servers clients on a
one-thread-per-client basis; it will receive a client request, process it and
return a reply (all in HTTP as described below). Since the App Engine distributes
each request to possibly a different server, concurrency can occur, so a form of
synchronization is needed. 

The datastore uses \emph{optimistic locking}\footnote{Optimistic locking (also
known as optimistic concurrency control) is based on the assumption that most
database transactions don't conflict with other transactions.} for concurrency
control. An update of an entity occurs in a transaction that is retried a fixed
number of times if other processes are trying to update the same entity
simultaneously. Our application can execute multiple datastore operations in a
single transaction, which either all succeed or all fail, ensuring the integrity
of data. 

Google provides us with an \emph{SQL} (Structured Query Language) alternative,
called the \emph{Google Query Language} (GQL). An example GQL query is shown in
figure \ref{gql-example}. These queries are rather powerful when using functions
like find(), as described below. Also, some security issues are triggered using
the GQL. Those are described in Section \ref{server-design-security}.

\begin{figure*}[ht] %[placement] where placement is h,t,b,p
\begin{center}
\begin{code}
db.GqlQuery("SELECT * FROM MyModel WHERE s1 >= :1 AND s2 < :2", "foo", "bar")
\end{code}
\caption{GQL Query example.\label{gql-example}}
\end{center}
\end{figure*}

\subsection{Client Functions}
The server provides some generic functions as an interface to its clients. These
functions adhere to the javadoc of the \emph{JavaGAT}\cite{javagat-www} (for
now). Below we will describe how these functions would be implemented. 

\subsubsection{HTTP Requests}
The Google App Engine only offers functions in a sandbox, as stated above. It is
not possible to make connections to the App Engine, other than using HTTP or
HTTPS. Furthermore, once a HTTP request is issued, a response is expected within
a few seconds, otherwise the connection times out. There is no specification on
the maximum size of an HTTP request/response.

Google offers an URL Fetch function, which supports five HTTP methods: GET, POST,
HEAD, PUT and DELETE. For the functions described below, it would be useful to
have an HTTP method that would allow us to send some (binary) data (i.e. POST or
GET), after which we do not only get a success return value; but also data as a
return value. This would be especially useful for the \texttt{find()} function.
The HTTP POST function servers our needs best, since the GET function translates
all data into a URL formatted String (which would be inappropriate for binary
data). Secondly, according to the HTTP RFC , the default HTTP response is a
status line (e.g. ``200 OK''), followed by a message body. This suits the needs
for the implementation of our server.

Our Java Adaptors could easily use these methods to authenticate themselves (with
or without Google Accounts), and call functions, possible sending data as
parameter of a function (using \emph{SOAP}\footnote{SOAP stands for Simple
Object Access Protocol, a protocol specification for exchanging structured
information in the implementation of Web Services. \cite{soap-www}} or
\emph{JSON} (JavaScript Object Notation) \cite{json-www} for example).

For Python \emph{simplejson} \cite{simplejson-www} is available, which is a
simple, fast, complete, correct and extensible JSON encoder and decoder for
Python 2.4+. It is pure Python code with no dependencies, but includes an
optional C extension for a serious speed boost. From Python 2.6, JSON is
contained in the standard library, but since Google makes use of Python version
2.5.2, it is likely that simplejson is needed.

\subsubsection{Functions to Implement} 
Looking at the javadoc of the JavaGAT \cite{javagat-javadoc}(the AdvertService in
specific), there are some basic functions we can implement using the App Engine.
First of all a very important function is the function called
\texttt{marshall()}, which takes a random object (stream of bytes) and creates a
String representation of this object. This object can then be used by the
following functions:

\begin{itemize}
\item \texttt{void add(MetaData, Object)}; Add an object with meta data to the
datastore
\item \texttt{void delete(path)}; Delete an object from the datastore
\item \texttt{String[] find(MetaData)}; Query the service for entries matching
a specified set of meta data
\item \texttt{Object getInstance(path)}; Gets an object from the datastore
\item \texttt{MetaData getMetaData(path)}; Gets meta data from the datastore
\item \texttt{String getPWD()}; Returns the current element of the name space
used
\item \texttt{void setPWD(path)}; Specify the element of the name space to be
used as reference for relative paths
\item \texttt{void exportDataBase(target URI)}; Imports the advert database
from persistent memory located at the given URI
\item \texttt{void importDataBase(source URI)}; Exports the advert database to
persistent memory located at the given URI 
\end{itemize}

Note that these functions are specific to the JavaGAT AdvertService, but could
also be used for the IPL registry bootstrap service, and possibly could be placed
somewhere in ibis.util for other use.

A good example are the \texttt{getPWD()} and \texttt{setPWD()} functions. Both
functions are necessary to make the AdvertService adapter for the Google App
Engine. However, these functions won't be very helpful for the other uses of our
service. That's why they won't be implemented into our library (which
communicates with the App Engine), but they will be in the adapter (locally).

\subsection{MetaData}
With the above in mind we also have to specify the MetaData object for other
services. Generally, a MetaData object consists of a number of key value tuples,
where both the keys and the values are Strings. A MetaData object should contain
some basic functionality:

\begin{itemize}
\item \texttt{String get(key)}; Gets the value associated to the provided key.
\item \texttt{String getData(int)}; Gets the value associated to the key retrieved by getKey(int).
\item \texttt{String getKey(int)}; Gets the i-th key of the MetaData.
\item \texttt{boolean match(MetaData)}; Match two MetaData objects.
\item \texttt{void put(key, value)}; Put an entry in the MetaData object.
\item \texttt{String remove(key)}; Removes an entry specified by the provided key.
\item \texttt{int size()}; Returns the number of entries in the MetaData.
\end{itemize}

Furthermore, the MetaData object implements \texttt{Serializable}, which means
the object can be converted into a sequence of bits so that it can be stored.

\subsection{Datastore Layout}
The App Engine scalable datastore stores and performs queries over data objects,
known as entities. An entity has one or more properties, named values of one of
several supported data types. A property can be a reference to another entity, to
create one-to-many or many-to-many relationships.

\subsubsection{Data Types Needed}
As we look at the javadoc from the AdvertService, we notice that we need to store
Advertisable objects in the Google Datastore, which is basically a key-value
pair. The client requests to store a sequence of bytes, which is marshaled by the
server and stored in the datastore.

The keys could be stored as a String, for example (as can be seen in the javadoc
of advert.MetaData). It would be nice to implement some sort of hierarchy for the
keys, if we consider that the server application could also be used as a
bootstrap server for the IPL registry. This could be implemented by the ``Key'',
provided by the Google Datastore.

The values are best stored as a string of bytes, since the client is allowed to
store any object in the datastore as wished. Google provides a data type for
these kind of data, which is called the BlobProperty . Blob is for binary data,
such as images. It takes a String value, but this value is stored as a byte
string and is not encoded as text, which is exactly what we need.

\subsubsection{Datastore Functions}
Besides traditional functions that Google provides to manipulate the datastore
(provided by the data modeling API, defined in \texttt{google.appengine.ext.db});
Google also provides an SQL like way to communicate with the datastore, called
the GQL.

\subsubsection{Garbage Collection}
Since the capacity of the datastore is limited to 500MB, we need some sort of
\emph{garbage collection} to keep our service usable for storing new data. This
is only possible by overwriting old data with new data (i.e. removing non-used
items to make room for new data). Removing old items is called garbage
collection.

\paragraph{TTL}
Initially we could give every data item stored in the server a Time to live
(TTL). This TTL could have a fixed value (for example 10 days), after which the
data will be removed from the datastore, making room for new data. Alternatively,
we could give data items a dynamic value for their TTL, depending on the usage of
the datastore (i.e. if the datastore is almost full, we will give data a shorter
TTL).

On every request (or once every x requests), we could query the datastore for
expired items and delete them from the datastore accordingly. Another option
would be that we remove the items which are expired as soon as we run out of free
storage space. This would give less overhead than starting a garbage collector
every request.

\paragraph{FIFO and LRU}
Secondly, if we would not give a TTL value to each data item stored in the
datastore, we could remove the oldest item as soon as we run out of free storage
space. By giving all items a timestamp as soon as we store then in the datastore,
we could find the oldest and remove it from the datastore to make room for new
data. Also we could make something like a LRU list (Least Recently Used), after
which we update the timestamp as soon as a data item was referenced or edited.
This way, only the data items that are least referenced are deleted, as soon as
we run out of storage.

\paragraph{Our Solution}
For now we would like to guarantee that a data item at least exists for a fixed
time span in the datastore. This way we prevent data items of being removed from
the datastore too quickly. Secondly the TTL should not be updated if a data item
is referenced. This could lead to clients referencing them only the keep them
alive in the datastore. Our main purpose for the TTL is to prevent clients from
storing data in the datastore and forgetting to delete it (which results in an
overfull datastore). Finally we won't need garbage collection every request, but
only when the datastore is reaching its limit (e.g. when 90\% is used). When the
datastore is full and there is no data to be evicted, we return an error to the
client, which throws an \texttt{AppEngineResourcesException}. The client can then
try again at a later moment in time, or use another server.

\subsection{Authentication and Privacy}
Google provides two forms of authentication with its App Engine.

\begin{itemize} 
\item By means of a Google Accounts (also used for Gmail, iGoogle, etc)
\item Google Apps for your Domain
\end{itemize}

For the first option, Google's unified sign-in system is used. All a user needs
is a valid email address (it doesn't need to be a Gmail address!) to sign up for
a Google Account. For the second option, users of Google Apps for your Domain can
choose to restrict all or part of their web application to only those people who
have a valid email address on their domain.

\subsubsection{Authentication through Google Accounts}
For our purposes it would be most attractive to make use of the authentication
through Google Accounts. An application can redirect a user to a Google Accounts
page to sign in or register for an account, or sign out; using simple functions
like \texttt{create\_login\_url()}. After a successful login session the
application can retrieve a User object to authenticate a user.

If one is to log in manually (\texttt{create\_login\_url()} was called), one
would see a login page, similar to that of other Google services, like Gmail for
example. Once one has pushed the `Sign in' button, username and password are sent
over HTTPS, after which a cookie is stored at the user side, containing a session
ID (ACSID). This cookie is used for all sessions until logout is initiated.
 
Note that this function is not very useful for our client, so we will have to
program the login process ourselves. There are solutions to interact with Google
data services, using cURL  for example .

\subsubsection{Own Authentication Scheme}
Second possibility would be to drop the concept of Google Account authentication,
and write our own authentication scheme (for example using a private-public key
pair). This has the advantage that we could apply a much more sophisticated
authentication scheme and give more guarantees with respect to the issues stated
above. A disadvantage is that it takes a lot more time and knowledge to implement
your own authentication scheme, while there is a perfectly fine solution at hand
(i.e. Google Accounts).

Another way of not using the authentication through Google accounts would be to
use the Django  framework, which is also supported by the Google Apps Engine.
Various examples of this authentication scheme exist.

\subsubsection{Server Authentication}
Both authentication schemes give us two options for server setups.

\begin{itemize}
  \item \textbf{Everyone Runs Its Own Service}; This way one could only
  authenticate himself by using a Google Account (i.e. one Google Account per
  organization), or restrict the application for only the domain of the
  organization using this service, or even to oneself. This way one would run out
  of resources less quickly than if one would use a (global) public service.
  Also, it is more reliable than using a public service; there is less chance of
  people reading and/or editing data stored by the service.
  \item \textbf{One Public Service for Everyone}; This way we would run an open
  service, accessible for everyone, without the need for a Google Account to
  authenticate oneself. A possible downside of this is that everyone can
  read/edit everyone's information. Also global public use might make the service
  run out of resources fast. Thus, for one global public service we can't give
  many guarantees.
\end{itemize}

\subsubsection{Conclusion}
Note that both solutions do not necessarily implement hierarchy in users. There
will be an interface for the administrator, but that stands aside of the service
we are offering (this interface is offered by the Google App Engine). Otherwise
it seems logical to use the Google Accounts for authentication (if we want to run
a 'private' service). For now we will implement two different servers; one
without any form of authentication (i.e. a public server), and one with
authentication (i.e. only the owner is able to use it, and additional users can
be added after the server is deployed).

\subsection{Security}
\label{server-design-security}
Finally we should take a look at security besides user authentication. More
decisions should be made as described below.

\subsubsection{(In)Secure Connections}
Google provides the means of having both insecure connections (HTTP, port 80), as
having secure connections (HTTPS, port 443). Naturally some data we would rather
not share in the open (for example passwords), so that would be a use for HTTPS.
The question is, if it would be wise (and necessary) to run all traffic over
HTTPS. We have to keep in mind that our bandwidth for HTTPS is limited (five
times less than our HTTP bandwidth), but is still a reasonable amount (see
Section \ref{appengine-quotas}). Plus we don't give any guarantees. If we run
out of HTTPS, we just have to wait another 24 hours and it will work again. Either way
configuration is done beforehand , when it comes to secure connections.

\subsubsection{GQL Queries}
\dashline{
SQL queries are generally vulnerable to SQL injection,
which is a technique that exploits a security vulnerability occurring in the
database layer of an application. Basically it would make it possible to insert
an SQL statement within a statement, and execute custom code. Usually this
custom code exists of malicious statements like \texttt{DROP}. Since the Google
App Engine's GQL Syntax only supports the \texttt{SELECT} stament, we do not
have to prevent this (see Figure \ref{gql-syntax}).
}

\begin{figure*}[ht] %[placement] where placement is h,t,b,p
\begin{center}
\begin{code}
  SELECT * FROM <kind>
    [WHERE <condition> [AND <condition> ...]]
    [ORDER BY <property> [ASC | DESC] [, <property> [ASC | DESC] ...]]
    [LIMIT [<offset>,]<count>]
    [OFFSET <offset>]

  <condition> := <property> {< | <= | > | >= | = | != } <value>
  <condition> := <property> IN <list>
  <condition> := ANCESTOR IS <entity or key>
\end{code}
\caption{GQL Syntax.\label{gql-syntax}}
\end{center}
\end{figure*}

\dashline{
In addition, the App Engine does not support multiple queries seperated by a
semi-colon (``\texttt{;}''), because it raises a parse error. Only one query at
a time can be executed and there is no support for nested queries also.
}
