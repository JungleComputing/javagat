\section{The App Engine}
\ldots

\subsection{The Runtime Environment}
\dashline{
As of April 8th 2009 \cite{app-engine-java} the Google App Engine provides two
programming languages to build and run web applications. Besides \emph{Python}
\cite{python-www}, also the \emph{Java} \cite{java-www} programming language is
featured to develop web applications and run them on Google's resources. Since
the feature of running your applications in Java is still relatively new, we
focused on writing our server in Python.
}

\dashline{
A Python web application interacts with the App Engine web server using the CGI
protocol. An application can use a WSGI-compatible web application framework
using a CGI adapter. App Engine includes a simple web application framework,
called \emph{webapp}, to make it easy to get started. For larger applications,
mature third-party frameworks, such as \emph{Django} \cite{django-www}, work
well with App Engine.
}

\dashline{
The App Engine currently only supports Python 2.5.3. The Python interpreter runs
in a secured ``sandbox'' environment to isolate your application for service and
security. The interpreter can run any Python code, including Python modules you
include with your application, as well as the Python standard library. The
interpreter cannot load Python modules with C code; it is a ``pure'' Python
environment.
}

\subsection{Limitations}
Although the Google App Engine provides us with resources, it also has some
limitations we should consider. All applications hosted on the App Engine
servers, are run within a sandbox. An application can only access other computers
on the Internet through the provided URL fetch and email services and APIs. Other
computers can only connect to the application by making HTTP (or HTTPS) requests
on the standard ports. An application cannot write to the file system. An app can
read files, but only files uploaded with the application code. The app must use
the App Engine \emph{datastore} for all data that persists between requests.
Application code only runs in response to a web request, and must return response
data within thirty seconds. A request handler cannot spawn a sub-process or
execute code after the response has been sent.

\subsection{Quotas}
\label{appengine-quotas}
The Google App Engine is free of use up to a certain level of used resources,
after which fees are charged for additional storage, bandwidth or CPU cycles
required by the application. Currently the following quotas, for using the free
version of the Google App Engine, are maintained:

\begin{itemize}
\item \textbf{CPU Time} - 46.30 hours/24h
\item \textbf{Outgoing Bandwidth} - 10.00 Gbytes/24h
\item \textbf{Incoming Bandwidth} - 10.00 Gbytes/24h
\item \textbf{Data Storage} - 1.00 Gbytes
\item \textbf{Secure Outgoing Bandwidth} - 2.00 Gbytes/24h
\item \textbf{Secure Incoming Bandwidth} - 2.00 Gbytes/24h
\item \textbf{Request size} - 10 MB per request
\item \textbf{Response time} - 30 seconds per request
\item \textbf{Data Sent to datastore} - 12.00 Gbytes/24h
\item \textbf{Data Received from datastore} - 116.00 Gbytes /24h
\item \textbf{Datastore CPU Time} - 62.11 hours/24h
\end{itemize}
From May 25th, new quotas will take effect \cite{app-engine-quotas}, which are:
\begin{itemize} 
\item \textbf{CPU Time} - 6.5 hours/24h (\$0.10 for every additional hour)
\item \textbf{Bandwidth} - 1 Gbyte (in and out)/24h (\$0.10/\$0.12 for every additional
GByte in/out)
\item \textbf{Stored Data \& Email Recipients} - these quotas will remain
unchanged (\$0.15 for every additional GB stored per month)
\end{itemize}

\subsection{Monitoring}
For us to monitor the quota already used by the various Ibis services, we can
login to the administration panel. All quotas as described above can be
monitored from this website, and additionally, we can see when our quotas will be
reset (being midnight, Pacific time). Historical note: The 24-hour replenishment
cycle was introduced in December 2008. It replaced a more complicated system of
``continuous'' replenishment, to make it easier to report and control resource
usage. Besides the monitoring of quotas, also various logs are available (e.g.
error logs, debug logs, and access logs).
