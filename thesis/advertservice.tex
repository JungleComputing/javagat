\section{AdvertService Adaptor}
\dashline{
Our first implementation of our Advert server is probably the most obvious use
of the Advert server; the \emph{JavaGAT AdvertService}. The AdvertService
allows \emph{Advertisable} instances to get published to and queried in an
advert directory. Such an advert directory is a meta data directory with an
hierarchical namespace attached \cite{javagat-javadoc}.
}

\subsection{Adaptor Implementation}
\dashline{
For the actual implementation of our \emph{AppEngineAdvertServiceAdaptor}, we
basically implemented all functionality provided by the JavaGAT API
\cite{javagat-javadoc}. This makes our implementation very similar to the
\emph{GenericAdvertServiceAdaptor}, with the major difference that we do not
use a local database for storage. Instead, we make use of the \emph{Ibis Advert
Client} library, which communicates with the App Engine (see Section 
\ref{clientimpl}). 
}

\dashline{
In the adaptor's constructor we connect to the App Engine by creating a new
\texttt{Advert} object (which in turn connects to the App Engine using the
\texttt{Communications} class). To connect to the authenticated version of the
Advert Service, username and password are required. Our adaptor attempts to 
fetch these from the \texttt{GATContext}. If none is found, we could either
decide to either connect to a public server or throw an exception to the user. 
}

\dashline{
Because of the fact that we do not implement our database locally, we cannot
implement two of the public functions given by the API, called
\texttt{importDataBase} and \texttt{exportDataBase}. Both functions are not
feasible to implement, since this would mean we theoratically should be able to
import/export 1 Gigabyte of data over HTTP, after which we would run out of
free quota too fast.
}