\section{IPL Registry Bootstrap}
\label{ipl}
Besides implementing a AdvertService adaptor for the Google App Engine, we
found another use for our advert service library, within Ibis. Our advert
server could very well function as a bootstrap server for an \texttt{IPL} (Ibis
Portability Layer) \cite{ipl-www} registry.

\subsection{IPL Server}
The Ibis Portability layer is a communication library is specifically
designed for usage in a grid environment. It has a number of properties which
help to achieve its goal of providing programmers with an easy to use, reliable
grid communication infrastructure \cite{ipl-www}.

A central concept in Ibis is the \emph{pool}. A pool consists of one or more Ibis
instances, usually running on different machines. Each pool is generally made up
of Ibises running a single distributed application. Ibises in a pool can
communicate with each other, and, using the registry mechanism present in Ibis,
can search for other Ibises in the same pool, get notified of Ibises joining the
pool, etc. To coordinate Ibis pools a socalled \emph{Ibis server} is used
\cite{ipl-usersguide}.

Services can be dynamically added to the server. By default, the Ibis
communication library comes with a \emph{registry service}. This registry
service manages pools, possibly multiple pools at the same time. Before starting
an Ibis application, an Ibis server needs to be running on a machine that is
accessible from all nodes participating in the Ibis run. Starting such
application requires the user to specify the server's address to the
application, in order to join a pool. Specifying such address can be well done
by means of a \emph{Bootstrap Service}, in which addresses are specified,
described by an identifier and (optional) meta data.

\subsection{Bootstrap Service Design}
Mapping our requirements of an IPL Registry Bootstrap service on our existing
Advert service, we have to make some design decisions. First of all, the user
should be able to decide whether a service is to be advertised or not. This can
be specified once the server is started, by using an extra argument when calling
\texttt{ipl-server}. For example, we could add \texttt{--advert ADVERT\_URL} as
parameter, after which we store the server's address at a certain advert service
, under a specific identifier (indicated by \texttt{ADVERT\_URL}). We structured
the \texttt{ADVERT\_URL} as follows:

\begin{center}
\begin{code}
aeas://servername.appspot.com/some_identifier
\end{code}
\end{center}

In this schema \texttt{aeas://} refers to an App Engine Advert Service. The
server needs to run on Google's \texttt{appspot.com} domain. Everyting after
the trailing \texttt{/} is considered the identifier at which the IPL Registry
is stored.

Secondly, we have to find a useful way to make use of meta data. Storing
key-value pairs can be really useful, if structured properly. Some examples
would be an \texttt{author} field (the user that started the IPL server), a
field with some sort of \texttt{time-of-creation} (supplying the time the
server was created), etc. We still need a means to properly structure
and pass meta data as soon as the server is created. The easiest way is to pass
an additional parameter like \texttt{--metadata}, after which we specify our
meta data as follows:

\begin{center}
\begin{code}
key1:value1,key2:value2,key3:value3
\end{code}
\end{center}

This way, our meta data is almost specified as it would be in JSON, which makes
it relatively easy to convert to a \texttt{MetaData} Object. No spaces or
special characters are allowed. Optionally we could implement meta data being
present in a separate file, which would require a file parser of some kind.
This is a future feature to implement.

Once a server is started using the \texttt{--advert} suffix, the server is added
to the Advert server (using the \texttt{add()} functionality already present).
Once a server is stopped, it will be removed (using the \texttt{del()}
function).

\subsection{Ibis Application Design}
Subsequently, when we start an Ibis application to join a certain pool, we
could not-specify \texttt{-Dibis.server.address}, but specify something like
\texttt{-Dibis.advert.address} instead. Also we have to add properties
like \texttt{-Dibis.advert.id} and \texttt{-Dibis.advert.md}, in order to
locate a running IPL server.

To be continued\ldots

